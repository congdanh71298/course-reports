
\title{Research Summary}
%\date{\today}

\documentclass[12pt]{report}
\usepackage{amsmath,amsxtra,amssymb,latexsym, amscd,amsthm}
\usepackage{url}
\usepackage{color}
\usepackage[mathscr]{eucal}
\usepackage{amsfonts}
\usepackage{graphicx}
\usepackage{fancybox}
\usepackage{multirow}
\usepackage{multicol}
\usepackage{array}
\usepackage[T1]{fontenc}
\usepackage[utf8]{inputenc}
\usepackage[english,vietnamese]{babel}
\usepackage[autostyle]{csquotes}
\usepackage{balance}
\usepackage[nottoc]{tocbibind}
\usepackage{float}
\usepackage{geometry}
\usepackage{lipsum}
\geometry{ a4paper, total={210mm,297mm},  left=20mm, right=20mm,  top=20mm, bottom=20mm }

\newcolumntype{L}[1]{>{\raggedright\let\newline\\\arraybackslash\hspace{0pt}}m{#1}}
\newcolumntype{C}[1]{>{\centering\let\newline\\\arraybackslash\hspace{0pt}}m{#1}}
\newcolumntype{R}[1]{>{\raggedleft\let\newline\\\arraybackslash\hspace{0pt}}m{#1}}
\newcommand\tab[1][1cm]{\hspace*{#1}}
\renewcommand{\bibname}{References}

\begin{document}

\thispagestyle{empty}
\begin{center}

	\vspace*{3cm}
	{\bf \LARGE BÁO CÁO ĐỒ ÁN}\\
	\vspace*{2cm}
	Tên đề tài:\\
	{\bf \Large Open-Set Grounded Text-to-Image Generation}\\
	\vspace{3cm}
	{\Large Trường Đại Học Công Nghệ Thông Tin}\\
	\vspace{5cm}
	%{\Large DOAN DUY}\\
	{\Large Tháng 8/2024}
\end{center}
%\thispagestyle{empty}

\newpage
\vspace*{5cm}
\begin{center}
	{\bf \Large Thông Tin Học Viên}\\
	Tên HV: \tab Mã HV:\\
	Nguyễn Công Danh \tab 230101033\\
	\vspace{5cm}
	{\bf \Large Giảng Viên Hướng Dẫn}\\
	TS. Mai Tiến Dũng
\end{center}

%	\frontmatter
\newpage
%%%%%%%%%%%%%%%%%%%%%%%%%%%%%%%%%%%%%%%%%%%
%+ Acknowledgment
%+ Revised: 13-07-2016
%+ Last revised: 24-07-2016
%%%%%%%%%%%%%%%%%%%%%%%%%%%%%%%%%%%%%%%%%%%
\section*{\centering{Acknowledgments}}
%\addcontentsline{toc}{chapter}{Acknowledgements}
Here is your acknowledgment.
Please giving your thanking to all people who have support or contribution to your work.


\begin{flushright}
Here is your name.
\end{flushright}

%%%%%%%%%%%%%%%%%%%%%%%%%%%%%%%%%%%%%%%%%%%
%+ End of Acknowledgment
%%%%%%%%%%%%%%%%%%%%%%%%%%%%%%%%%%%%%%%%%%%

\tableofcontents
\listoffigures
\listoftables
%\pagenumbering{arabic}
\chapter{Giới thiệu}
\section{Lý do chọn đề tài}
Tiên phong trong lĩnh vực nghiên cứu:
Đề tài này liên quan đến một phương pháp mới có tên là GLIGEN,
cho phép tạo ra hình ảnh từ văn bản có tính năng kiểm soát cao.
Phương pháp này mở rộng khả năng của các mô hình trước đó bằng cách cho phép sử dụng các thông tin đầu vào
bổ sung như bounding box, keypoints, canny map, depth map, normal map, hed map và semantic map.

Đóng góp lớn cho cộng đồng học thuật:
GLIGEN đã được đánh giá và chứng minh có khả năng vượt trội so với các mô hình trước đây
trong việc tạo hình ảnh dựa trên văn bản ở các nhiệm vụ mới và đa dạng,
điều này giúp đóng góp lớn cho cộng đồng nghiên cứu trong việc nâng cao chất lượng
và độ chính xác của các hệ thống tạo hình ảnh.

Tính ứng dụng cao: Phương pháp GLIGEN không chỉ cải thiện về mặt lý thuyết
mà còn có tiềm năng ứng dụng rộng rãi trong nhiều lĩnh vực như thiết kế đồ họa
và tạo nội dung số.

\section{Phát biểu bài toán}

Bài toán trong bài báo này được phát biểu như sau:
\begin{itemize}
	\item Input: Đầu vào bao gồm một chú thích văn bản (caption)
	      và các thông tin định vị (grounding information) như hộp giới hạn (bounding boxes),
	      điểm mốc (keypoints), có thể được mô tả bằng văn bản hoặc hình ảnh tham chiếu.
	\item Output: Đầu ra là hình ảnh được sinh ra dựa trên chú thích văn bản
	      và các thông tin định vị, đảm bảo rằng hình ảnh này thể hiện chính xác các yếu tố
	      và vị trí của chúng theo đầu vào đã cung cấp.
\end{itemize}

\section{Kết quả mong đợi}
Kết quả mong đợi khi thực hiện đề tài này là tìm hiểu một phương pháp mới
có khả năng tạo ra hình ảnh từ văn bản và thông tin định vị với độ chính xác cao.

\section{Phạm vi/Giới hạn}
Khi thực hiện đề tài này, em sẽ tập trung vào việc tìm hiểu phương pháp,
cách cài đặt và sử dụng lại mô hình có sẵn.
Không đi sâu vào việc nghiên cứu và phát triển mô hình mới.

\chapter{Phương pháp}



\newpage
\begin{thebibliography}{1}
	\bibitem{ref1}
	Li, Y., Liu, H., Wu, Q., Mu, F., Yang, J., Gao, J., Li, C., \& Lee, Y. J. (2023). GLIGEN: Open-Set Grounded Text-to-Image Generation. In Proceedings of the IEEE/CVF Conference on Computer Vision and Pattern Recognition (CVPR) (pp. 22511-22521).
\end{thebibliography}
\end{document}